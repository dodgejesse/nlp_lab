\documentclass[11pt,letterpaper]{article}
\usepackage{fullpage}
\usepackage[pdftex]{graphicx}
\usepackage{amsfonts,eucal,amsbsy,amsopn,amsmath}
\usepackage{url}
\usepackage[sort&compress]{natbib}
\usepackage{natbibspacing}
\usepackage{latexsym}
\usepackage{wasysym} 
\usepackage{rotating}
\usepackage{fancyhdr}
\DeclareMathOperator*{\argmax}{argmax}
\DeclareMathOperator*{\argmin}{argmin}
\usepackage{sectsty}
\usepackage[dvipsnames,usenames]{color}
\usepackage{multicol}
\definecolor{orange}{rgb}{1,0.5,0}
\usepackage{multirow}
\usepackage{sidecap}
\usepackage{caption}
\renewcommand{\captionfont}{\small}
\setlength{\oddsidemargin}{-0.04cm}
\setlength{\textwidth}{16.59cm}
\setlength{\topmargin}{-0.04cm}
\setlength{\headheight}{0in}
\setlength{\headsep}{0in}
\setlength{\textheight}{22.94cm}
\allsectionsfont{\normalsize}
\newcommand{\ignore}[1]{}
\newenvironment{enumeratesquish}{\begin{list}{\addtocounter{enumi}{1}\arabic{enumi}.}{\setlength{\itemsep}{-0.25em}\setlength{\leftmargin}{1em}\addtolength{\leftmargin}{\labelsep}}}{\end{list}}
\newenvironment{itemizesquish}{\begin{list}{\setcounter{enumi}{0}\labelitemi}{\setlength{\itemsep}{-0.25em}\setlength{\labelwidth}{0.5em}\setlength{\leftmargin}{\labelwidth}\addtolength{\leftmargin}{\labelsep}}}{\end{list}}

\bibpunct{(}{)}{;}{a}{,}{,}
\newcommand{\nascomment}[1]{\textcolor{blue}{\textbf{[#1 --NAS]}}}


\pagestyle{fancy}
\lhead{}
\chead{}
\rhead{}
\lfoot{}
\cfoot{\thepage~of \pageref{lastpage}}
\rfoot{}
\renewcommand{\headrulewidth}{0pt}
\renewcommand{\footrulewidth}{0pt}


\title{11-712:  NLP Lab Report}
\author{Jesse Dodge}
\date{April 25, 2014}

\begin{document}
\maketitle
\begin{abstract}
\nascomment{one paragraph here summarizing what the paper is about}
\end{abstract}

\nascomment{brief introduction}

\section{Basic Information about French}
French is a Roman language, spoken in many countries around the world. It's the 10th most commonly spoken language in the world, with between 220 and 300 million people speaking French as a first or second language. French syntax is quite similar to English, with a few key differences. For example, most French adjectives go after the word they're modifying, the French negation has two parts, and determiners have gender. 

\section{Past Work on the Syntax of French}
French syntax has been well studied. There exist texts ranging from simple (e.g. Fundamentals of French Syntax\footnote{http://www.academia.edu/1997083/Fundamentals\_of\_French\_Syntax}) to complex (e.g. The Syntax of French\footnote{http://www.amazon.com/Syntax-French-Cambridge-Guides/dp/B008SLJ1WQ}, Foundations of French Syntax\footnote{http://www.amazon.com/Foundations-French-Cambridge-Textbooks-Linguistics/dp/0521388058}). Beyond just textbooks, there are a number of papers that address individual topics in French such as verb movement\footnote{http://minimalism.linguistics.arizona.edu/AMSA/PDF/AMSA-175-1000.pdf} and tokenization\footnote{https://giguete.users.greyc.fr/pricai96/part4.html}.

With repsect to syntax for parsing, there exists a French constituency-parsed treebank\footnote{http://alpage.inria.fr/statgram/frdep/fr_stat_dep_parsing.html} which contains 12,531 sentenecs from Le Monde, a French newspaper. This treebank has been annotated in a similar style to the Penn treebank, with a few additions. For example, each word (when applicable) contains information on gender. 

Note: we can put something more here about other syntax formalisms as they come up.


\section{Available Resources}
TreeTagger\footnote{http://www.cis.uni-muenchen.de/~schmid/tools/TreeTagger/} is an open-source POS tagger for French. Europarl\footnote{http://www.statmt.org/europarl/} has 194 MB copus of French-English text that I can annotate. Simply taking the first 2000 words and dividing it into A and B test sets will give us the data we need. Note: The fact this is parallel data is interesting, but not necessary. (Note: There exists a large corpus of Le Monde\footnote{http://alpage.inria.fr/statgram/frdep/fr_stat_dep_parsing.html} that I can use either as the test corpora as well, if Europarl is too formal. Also, this data can be used for semi-supervised training, though it is annonated already with constituency parses.) 

\section{Survey of Phenomena in \nascomment{Your Language/Genre}}

\section{Initial Design}

\section{System Analysis on Corpus A}

\section{Lessons Learned and Revised Design}

\section{System Analysis on Corpus B}

\section{Final Revisions}

\section{Future Work}





\bibliographystyle{plainnat}
\bibliography{refs}
\label{lastpage}
\end{document}
